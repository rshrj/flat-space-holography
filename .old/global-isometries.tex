\documentclass[11pt]{article}

\usepackage{amsmath, physics, amssymb, amsfonts, amsthm, mathtools, dsfont}
\usepackage{palatino}
\usepackage[margin=1in]{geometry}
\usepackage{titling}
\usepackage{graphicx, xcolor}
\usepackage[none]{hyphenat}
\usepackage{titlesec}
\usepackage[hidelinks]{hyperref}
\usepackage{import}
\usepackage{xifthen}
\usepackage{pdfpages}
\usepackage{transparent}
\usepackage{multicol}

\setlength{\parindent}{0in}
\setlength{\parskip}{0.1in}

\newcommand{\dMeas}[1]{\text{d} #1 \,}
\newcommand{\B}[1]{\boldsymbol{#1}}
\newcommand{\residue}[2][]{\underset{#1}{\text{Res}}\Big(#2\Big)}
\newcommand{\EuclideanOrder}{\text{\large T}_{\text{E}}}
\newcommand\Set[2]{\{\,#1\mid#2\,\}}

\numberwithin{equation}{section}

\titleformat{\section}[block]{\LARGE\bfseries}{}{0em}{}[]


% Title

\pretitle{\begin{flushright}\Huge\scshape}
\posttitle{\par\end{flushright}}

\preauthor{\begin{flushright}\large\bfseries}
\postauthor{\par\end{flushright}}

\predate{}
\postdate{}

\title{Isometries of Global \texorpdfstring{$AdS_3$}{TEXT}}
\author{Rishi Raj}
\date{}

\begin{document}
\raggedright
\maketitle

\section{\texorpdfstring{$\B{AdS_3}$}{TEXT}}
Consider the following subset of $\mathbb{R}^{2,2}$ (with coordinates $(X^{-1}, X^0, X^1, X^2) \equiv X^A$).
\begin{equation}\label{ads}
  AdS_3 \coloneqq \Set{X^A}{X^A X_A = -R^2} \hspace{0.03\linewidth} R \in \mathbb{R}^+ 
\end{equation}
It inherits topology, smooth structure and metric from $\mathbb{R}^{2,2}$ and is a Lorentzian 3-manifold.

It has the topology of $\mathbb{S}^1 \cross \mathbb{S}^1 \cross \mathbb{R}$ and has closed timelike curves, so one instead considers its universal cover which is topologically $\mathbb{S}^1 \cross \mathbb{R}^2$.

Note that the above subset with $R = 0$ considered as a subset of $\text{P}\mathbb{R}^5$ is just the conformal compactification of Minkowski space $\mathbb{R}^{1,1}$ to $(\mathbb{S}^1 \cross \mathbb{S}^1)/\B{Z}_2$. The same subset is also the conformal boundary of \ref{ads} for finite $R$ as seen by scaling $X^A = s \tilde{X}^A$ and taking $s \rightarrow \infty$.

\section{Coordinates and Metric}
One can introduce a coordinate chart in \ref{ads} by simply noting that
\begin{align}
  - (X^{-1})^2 - (X^{0})^2 + (X^{1})^2 + (X^{2})^2 = -R^2
\intertext{so that}
  (X^{-1})^2 + (X^{0})^2 = R^2 \cosh^2(\rho) \cr
  (X^{1})^2 + (X^{2})^2 = R^2 \sinh^2(\rho)
\end{align}
does the job and continuing the same thing, one gets the so-called \textbf{global coordinates} on $AdS_3$ (in hyperbolic form)
\begin{align}\label{globalcoordinates}
  X^{-1} &= R \cosh{\rho} \cos{\tau} \cr
  X^{0} &= R \cosh{\rho} \sin{\tau} \cr
  X^{1} &= R \sinh{\rho} \cos{\phi} \cr
  X^{2} &= R \sinh{\rho} \sin{\phi}
\end{align}

Writing the $\mathbb{R}^{2,2}$ metric in these coordinates gives the Global $AdS_3$ metric:
\begin{equation}\label{globalmetric}
  \text{d}s^2 = R^2 (-\cosh{\rho}^2 \text{d}\tau^2 + \text{d}\rho^2 + \sinh{\rho}^2 \text{d}\phi^2)
\end{equation}

\section{Isometries}
One can easily see that the isometries of \ref{ads} are those of $\mathbb{R}^{2,2}$ that leave the origin fixed. That is, the symmetry group is $\mathcal{SO}(2, 2)$. The embedding coordinates transform as:
\begin{gather}
  X^A \rightarrow Y^A \, = \, \Lambda^{A}_{\; B} \, X^{B}
  \intertext{with}
  \eta_{A B} \Lambda^{A}_{\; C} \Lambda^{B}_{\; D} = \eta_{C D}
\end{gather}
where $\eta = \text{diag}(-1, -1, 1, 1)$

This includes two rotations and four boosts. The rotations are the simplest as they are just $\tau$ and $\rho$ translations:
\begin{align}
  \text{$X^{-1}$-$X^{0}$ rotation} \hspace{0.05\linewidth}& \tau \rightarrow \tau^\prime = \tau + c \cr
  \text{$X^{1}$-$X^{2}$ rotation} \hspace{0.05\linewidth}& \phi \rightarrow \phi^\prime = \phi + d
\end{align}
Also note that these are unchanged when one takes the boundary limit ($\rho \rightarrow \infty$)

The boosts are a bit complicated but straightforward to derive. I write one and others follow from combinations of rotations above. E.g. the $X^{0}$-$X^{2}$ boost with parameter $\nu$
\begin{equation}
  \begin{pmatrix}
    X^{-1} \\
    X^{0} \\
    X^{1} \\
    X^{2}
  \end{pmatrix} \rightarrow
  \begin{pmatrix}
    Y^{-1} \\
    Y^{0} \\
    Y^{1} \\
    Y^{2}
  \end{pmatrix} = 
  \begin{pmatrix}
    1 & 0 & 0 & 0 \\
    0 & \cosh{\nu} & 0 & -\sinh{\nu} \\
    0 & 0 & 1 & 0 \\
    0 & -\sinh{\nu} & 0 & \cosh{\nu}
  \end{pmatrix} \begin{pmatrix}
    X^{-1} \\
    X^{0} \\
    X^{1} \\
    X^{2}
  \end{pmatrix}
\end{equation}
On the Global coordinates, this turns out to be
\begin{align}
  \rho \, \rightarrow& \, \rho^\prime = \cosh^{-1}\sqrt{(\cosh (\nu ) \cosh (\rho) \sin
  (\tau)-\sin (\phi) \sinh (\nu ) \sinh (\rho))^2+\cosh ^2(\rho) \cos
  ^2(\tau)} \cr
  \tau \, \rightarrow& \, \tau^\prime = \tan^{-1}(\cosh (\nu ) \tan (\tau) - \sin (\phi) \sinh (\nu ) \tanh
  (\rho) \sec (\tau)) \cr
  \phi \, \rightarrow& \, \phi^\prime = \tan^{-1}(\cosh (\nu ) \tan (\phi) - \sec (\phi) \sinh (\nu ) \coth (\rho) \sin (\tau))
\end{align}

On the boundary, these expressions simplify a bit
\begin{align}
  \rho \, \rightarrow& \, \rho^\prime = \frac{1}{2} \rho \ln((\cosh (\nu ) \sin
  (\tau)-\sin (\phi) \sinh (\nu ))^2 + \cos
  ^2(\tau)) \cr
  \tau \, \rightarrow& \, \tau^\prime = \tan^{-1}(\cosh (\nu ) \tan (\tau) - \sin (\phi) \sinh (\nu )\sec (\tau)) \cr
  \phi \, \rightarrow& \, \phi^\prime = \tan^{-1}(\cosh (\nu ) \tan (\phi) - \sec (\phi) \sinh (\nu )\sin (\tau))
\end{align}

\end{document}